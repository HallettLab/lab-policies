\documentclass[12pt]{article}
\usepackage[top=0.8in, bottom=0.8in, right=0.8in, left=0.8in,
paperwidth=8.5in, paperheight=11in, nohead]{geometry}
\geometry{letterpaper}
\usepackage[pdftex]{graphicx}
\usepackage{color}
\usepackage[normalem]{ulem}
\usepackage{amssymb}
\usepackage{amsmath}
\usepackage{epstopdf}
\usepackage{setspace}
\usepackage{mdwlist}
\usepackage{hyperref}

\title{Lab structure}
\author{Hallett Lab}

\begin{document}
\maketitle

\section{Lab meetings}
\begin{enumerate}
\item Lab meetings are held weekly and should be prioritized by everyone, with the exception of conferences or time-sensitive field work.
\item Lab meeting schedules are set at the beginning of each quarter; typically each person has a half or full meeting dedicated to their work. Typically we start by prioritizing any upcoming deadlines and fill in with other projects that could benefit from group attention (be it experimental design, results interpretation, manuscript review, or a practice presentation). 
\item Occasional lab meetings are dedicated to maintaining a smooth lab, such as updating data storage, reviewing lab policies, or setting six month plans. 
\end{enumerate}

\section{Meetings with Lauren}
\begin{enumerate}
\item Each graduate student/postdoc will have a scheduled standing meeting with Lauren. Typically these will be longer and more consistent early in a project, shorter and/or held on an as-need basis during active data collection, and more regular again during analysis, writing, and follow-up.
\item It is expected that you have an agenda prepared for the meeting and outline your goals for the meeting at the top (Lauren may add agenda items at this time too). 
\item Meetings should be listed on the lab calendar (with the time and your name). It is your responsibility to keep the calendar up-to-date with your individual and project meetings (if Lauren doesn't see it on the calendar, she will tend to forget it is happening).
\item First year graduate students will initially use this time as an informal one-on-one reading group with Lauren to explore the literature, hone your interests, and work toward potential thesis questions. Subsequently the meetings can be used to discuss experimental design, analysis, proposals and drafts, etc. 
\item Periodically these meetings should be used to discuss longer-term goals and professional development (generally it's best to hold these types of meetings when you first join the lab and after each six-month plan exercise) to make sure that everyone's goals are being met.
  \end{enumerate}

\section{Lab retreats}
\begin{enumerate}
\item Lab retreats are held at least annually and should be prioritized by grad students and postdocs.
\item Lab retreats are scheduled for times when many members have either proposals or manuscript drafts ready for larger group review (often July-August).
\item "Writing workshops" are a core component of lab retreat, with each person submitting either a draft or extended figures, and the group workshopping each in turn. See the "write workshop" guidelines for lab protocol on the structure of our in-person review process (we use this process both in lab meetings and retreats).
\item Other types of work activities in lab retreats can included outlining or working on lab papers, checking in on core aspects of lab functioning (e.g., data management, lab inventory), and reflecting on and discussing short and long-term goals. 
\item Lab retreats end with something fun! Ideally retreats are an actual retreat (we go somewhere interesting), but at a minimum should involve a lab bonding activity at the end.
\end{enumerate}

\section{NutNet group fieldwork}
\begin{enumerate}
\item The lab maintains two Nutrient Network field sites at the H J Andrews field station (Bunchgrass Meadow and Lookout Mt). Assisting with annual data collection should be prioritized by grads/postdocs and summer undergraduates.
\item NutNet sampling typically occurs over a two-three day period in mid to late July. The lab stays at the H J Andrews bunkhouse and conducts field sampling together (with group cooking/social activities in the evening).
\item Our goals for the NutNet sampling include: 1) Supporting and connecting to a larger network of ecologists, 2) Training newer members in field methods, 3) Building bonds among the lab, 4) Sustaining a relationship with local ecosystems. 
\item Lab members are encouraged to think of research questions that could be tested with either the local or larger NutNet dataset, and to work with Lauren to plug into the network. They are very welcoming!
\end{enumerate}


\section{Six month plans}
\begin{enumerate}
\item Six month plans are an opportunity to reflect on your short and long term goals (really they include six-month plans as well as one-year and five-year goals). You can use whatever structure works best for you - some people like to list items to accomplish and the anticipated time investment of each, others like to make calendars with sequential goals. 
\item Six month plans are discussed as a full group. It is really helpful for everyone to know what everyone else is up to for maximum collaboration and support. It is also helpful to get feedback on the realism of different plans (e.g., how long will it take you to sort biomass? Someone else in the lab will be able to advise). 
\item Six month plans are also a way for both you and Lauren to make sure you are progressing in a timely way. Unforseen opportunities and obstacles may arise over six months, but as a general rule assessing your progress against your six month plan is a good way to take stock of what is and is not working, and to identify paths to address it.
\end{enumerate}


\section{Submitting grants}
\begin{enumerate}
\item Typically, per university policy, Lauren is a PI on all grants originating from
  the lab (excluding fellowships and ``award'' style grants aimed at grads/postdocs. Postdocs can be PIs with some funding agencies, however; if this applies Lauren will work with you to make sure you can get official credit as a PI or co-PI.
\item All grants must be approved by the sponsored programs office. They need \textit{at least} 3 business days before the grant is due to approve the grant. It is recommended that you get materials uploaded well in advance. It is most important to have the budget and budget recommendation uploaded with a good time buffer; the project narrative can be in draft form until closer to the final deadline.  
\item We have an assigned grant officer.  Address all questions regarding budgets (i.e., the graduate student step system for stipends) to him/her. Start working on budget one month prior to grant submission.
\item Please be courteous with the IEE and sponsored projects grants staff! They have to deal with stressed out people a lot, let's make their lives easier by being polite in our requests for help and timely with our grant item uploads.
\end{enumerate}

\section{Requesting comments}
\begin{enumerate}
\item For manuscripts, be sure to storyboard the manuscript with
  Lauren before any serious writing.
\item Have a peer (i.e., post-doc or graduate student in the lab or
  with similar research interests) read the draft before submitting it
  to Lauren.
\item \textbf Expect review to take two weeks or more depending on how many manuscripts are on Lauren's queue to review.
\end{enumerate}

\section{Submitting manuscripts}
\begin{enumerate}
\item All coauthors must read and sign off on a manuscript before
  submission. Online submission is done by the first author.
\item Before submission, re run all code and check the reported statistics in
  manuscript.
\item Before publication (and ideally before submission), check that your workflow is public and interpretable (e.g., well commented and with the order of analyses well explained. For complicated workflows, an explanation in the README is useful; it can also be helpful to break the code into files associated with each major analysis or figure.
\item Share communication from the journal with your coauthors in a timely way (e.g., reviews, acceptances or rejections).
\item When responding to reviews, include a cover letter that explains major changes, a line-item response to each comment, and either a tc or latex diff file of the manuscript. It can be very helpful to have a meeting with coauthors upon receiving a review to discuss a revision plan (rather than stewing with it alone and only asking for feedback after a first pass), and then sending the full revised package back out after completion. Only resubmit after all have responded (either with comments or by confirming they don't have comments). If an external coauthor is particularly difficult to reach, it can be helpful to give them a fair deadline (e.g., "if I do not hear from you in the next two weeks I will assume you do not have comments and are fine with submission").
\end{enumerate}

\section{Signatures}
Either bring hand copies that need signatures to Lauren during your weekly meeting (or by dropping by her office), or leave them in Lauren's IEE mailbox  with a tab marking the line the signature is needed. If you go the latter route, drop Lauren an email so she knows to look for it! Otherwise, email Lauren items that can have electronic signatures (for rapid turnaround, it's helpful to put "signature needed" in the subject line).

\section{Requesting letters of recommendation}
Please give Lauren a good lead time on letters of recommendation (ideally at least one month in advance). Let her know what it is for, when it is due, and detailed instructions on how it needs to be submitted (the email, the link, etc.). It is also very useful to follow up with your latest CV and some talking points you would like to have highlighted. \textbf{Please remind Lauren of deadlines/ask for submission confirmation as the deadline approaches.} Also, if you are requesting a batch of letters (e.g., for graduate school or faculty jobs), it is really helpful to provide a google spreadsheet of each application, its deadline, and its submission method so that one doesn't get missed.
\end{document}

%%% Local Variables:
%%% mode: latex
%%% TeX-PDF-mode: t
%%% End:


