\documentclass[12pt]{article}
\usepackage[top=0.8in, bottom=0.8in, right=0.8in, left=0.8in,
paperwidth=8.5in, paperheight=11in, nohead]{geometry}
\geometry{letterpaper}
\usepackage[pdftex]{graphicx}
\usepackage{color}
\usepackage[normalem]{ulem}
\usepackage{amssymb}
\usepackage{amsmath}
\usepackage{epstopdf}
\usepackage{setspace}
\usepackage{mdwlist}
\usepackage{hyperref}
\usepackage{xr}


\title{Personnel}
\author{Hallett Lab}

\begin{document}

\maketitle

\section{PI - What you can expect of Lauren}
\begin{enumerate}
\item Check in on and forward the professional interests of the members of the lab, including expanding your professional networks.
\item Foster and encourage an atmosphere of inclusivity, collaboration, exploration,
  learning, and good communication.
\item Provide mentorship on research projects, from experimental design through to publication, and mentorship on collaborations and career trajectories.
\item Be aware of your progress and provide structure to keep you on track, including feedback when things are going well and shining light on areas for growth.
\item Be timely in feedback on abstracts and manuscripts (typically within a 2 week turn around).
\item Be timely and communicative when writing letters of recommendation.
\item Ultimately be responsible for research funding, accounting,
  scientific oversight, and training and oversight of personnel.
\item Ensure that public dollars are
  well-spent; that the science that emerges from the lab is productive,
  rigorous, and cutting edge.
\item I must approve all abstracts, manuscripts, or any representation
  of the research that comes from the lab, before it leaves the door
  for approval by collaborators or submission to meetings or journals.
\end{enumerate}


\section{Graduate Students - What Lauren expects of you}
\begin{enumerate}
\item Be a leader in the laboratory and contribute to an inclusive and collegial environment. This includes:
\begin{enumerate}
\item Actively reaching out and welcoming new members
\item Supporting one another in the field 
\item Paying it forward in terms of teaching others protocols and methods
\item Providing feedback on each other's experimental design, proposals, talks, and drafts
\item Attending and engaging in lab functions (lab meetings, lab retreats, etc) 
\item Treating one another with courtesy and respect
\end{enumerate}

\item Set your own schedule and communicate and develop your goals with Lauren; this is best done through six month plans. We will use these plans to make sure we are on the same page about both your research and professional goals and to benchmark your progress.

\item Develop your own research ideas; this involves spending time with the literature, deep thinking and brainstorming, and being receptive and responsive to feedback from Lauren and the lab.

\item Request feedback on abstracts, proposals, and drafts in a timely way (ideally by providing the document with plenty of lead time; during crunch periods by communicating with Lauren about when she can anticipate the document). It is essential that any documents with outside coauthors be shared with Lauren with sufficient lead time to then be shared with coauthors for their feedback and approval.

\item Take initiative in identifying additional mentors and collaborators as needed and discussing these options with Lauren, who will provide guidance in how to reach out and expand your mentor network.

\item Be flexible, available and willing to put in the time and work in emergencies or crunch times (particularly field season).

\item Make an effort to fund your own research by seeking out fellowship and grant opportunities, responding when Lauren directs you to funding opportunities, and putting care and attention into proposals. You may also be asked to contribute to larger lab proposals (e.g., by sharing preliminary data or providing edits).
  
\item Mentor undergraduate researchers. This involves both overseeing undergrad helpers in the lab, and serving as co-advisors on senior theses for undergrads who exhibit interest and promise in engaging in the lab more substantively. Mentoring thesis students typically involves:
\begin{enumerate}
\item Working with Lauren and the undergrad to identify a thesis project. Ideally this will involve carving off a subproject that is related to and supports your own work.
\item Providing the undergrad with training on protocols and methods.
\item Meeting with the undergrad to discuss the literature and ensure they understand the ``why'' as well as the ``what'' of what they are doing.
\item Directing the undergrad to funding opportunities (e.g., UROP, SPUR, O'Day) and providing feedback on their proposals.
\item Workshoping thesis drafts and presentations with the student.
\item Providing guidance for students on next steps after graduation (e.g., providing feedback on grad school inquiries/applications for inclined students).
\item Keeping track of highlight points and examples for letters of recommendation, and working with Lauren to write these in a timely way.
\item Communicating with Lauren about the process and seeking her guidance if undergrads are not maintaining their end of the agreement.
\end{enumerate}

\item Contribute to assigned lab tasks (keeping website up-to-date, tidying up, maintaining the listserv or lab meeting schedule, etc.)

\item Attend research conferences, especially in the last few years of
  your tenure. Lab funds for travel are limited non-grant related work, so you must try to
  fund yourself through grants from the conference and department.
  
\item Be present during the academic school year, except for fieldwork, conference travel, etc. Lauren will not keep track of when you are in the lab or office, but you are expected to be present enough for the casual interactions and discussions that fuel collaboration with your labmates and colleagues.  

Grad school is a long time, and life happens. Lauren will work with you to find a balance between personal needs and professional progress. If you anticipate a reason to be away from the lab for an extended time, communicate with Lauren to work out a plan together. Note that flexibility here will depend on a number of factors, including the need, the available funding, and your ability to engage remotely.
  
If you are away for any reason (field work, personal time, etc) you are expected to note it on the lab calendar so that folks are not wondering where you are. 

\item Follow lab protocols such as equipment checkout, data and project management, lab safety, etc.

\end{enumerate}

\textit{Note that these expectations apply to both MS and PhD students, but to different degrees. For example, while PhD students are expected develop their own research programs, MS students may be more likely to plug into existing projects. Similarly, PhD students will often mentor teams of undergraduates, whereas MS students are likely to work with just one or two. Lauren will try to balance requests for lab task contributions against other insitutional demands on your schedule (e.g., teaching and courseload requirements), acknowledging that MS and early PhD students typically have substantial other institutional demands on their time.}



\section{Postdocs - What Lauren expects of you}
\begin{enumerate}
\item Be a leader in the laboratory and contribute to an inclusive and collegial environment. This includes:
\begin{enumerate}
\item Actively reaching out and welcoming new members
\item Supporting one another in the field 
\item Paying it forward in terms of teaching others protocols and methods
\item Providing feedback on each other's experimental design, proposals, talks, and drafts
\item Attending and engaging in lab functions (lab meetings, lab retreats, etc) 
\item Treating one another with courtesy and respect
\end{enumerate}

\item Set your own schedule and communicate and develop your goals with Lauren; this is best done through six month plans. We will use these plans to make sure we are on the same page about both your research and professional goals and to benchmark your progress.

\item Communicate with Lauren about projects outside of the lab, such as PhD follow-up (e.g., include it on your six month plan and discuss in your weekly meeting when it takes some of the week's time). While the majority of your effort should be focused on your current project, it is important to be able to finish previous work. In general good communication is enough to find the right balance here. Note that there is greater flexibility for postdocs on their own fellowship funding than on grant-funded projects, although in all cases you are responsible for the scope of work outlined in your project.

\item Develop your own research ideas; this involves spending time with the literature, deep thinking and brainstorming, and being receptive and responsive to feedback from Lauren and the lab.

\item Request feedback on abstracts, proposals, and drafts in a timely way (ideally by providing the document with plenty of lead time; during crunch periods by communicating with Lauren about when she can anticipate the document). It is essential that any documents with outside coauthors be shared with Lauren with sufficient lead time to then be shared with coauthors for their feedback and approval.

\item Take initiative in opportunities to expand your professional network in communication with Lauren. This may involve writing synthesis proposals, presenting talks, developing your professional website, or cultivating new collaborations. 

\item Be flexible, available and willing to put in the time and work in emergencies or crunch times (particularly field season).

\item Anticipate and seek out your own funding via fellowships and research/travel grants in communication with Lauren. Depending on your interest, Lauren will also work with you to write your own larger grants. You may also be asked to contribute to larger lab proposals (e.g., by sharing preliminary data or providing edits).

\item Work with Lauren to manage your project's budget and assist with annual project reports.

\item Identify and pursue projects that expand or complement your core research, so long as the project you are hired to contribute to is progressing on schedule. The lab philosophy is that you have put a lot of time into developing independent research skills during your PhD - let's use it to expand the scope and quality of science in the lab.

\item Mentor undergraduate researchers. This involves both overseeing undergrad helpers in the lab, and serving as co-advisors on senior theses for undergrads who exhibit interest and promise in engaging in the lab more substantively. Mentoring thesis students typically involves:
\begin{enumerate}
\item Working with Lauren and the undergrad to identify a thesis project. Ideally this will involve carving off a subproject that is related to and supports your own work.
\item Providing the undergrad with training on protocols and methods.
\item Meeting with the undergrad to discuss the literature and ensure they understand the ``why'' as well as the ``what'' of what they are doing.
\item Directing the undergrad to funding opportunities (e.g., UROP, SPUR, O'Day) and providing feedback on their proposals.
\item Workshoping thesis drafts and presentations with the student.
\item Providing guidance for students on next steps after graduation (e.g., providing feedback on grad school inquiries/applications for inclined students).
\item Keeping track of highlight points and examples for letters of recommendation, and working with Lauren to write these in a timely way.
\item Communicating with Lauren about the process and seeking her guidance if undergrads are not maintaining their end of the agreement.
\end{enumerate}

\item Mentor graduate researchers. This involves a collaborative relationship with graduate students that overlap with you in either study system or conceptual questions. Lauren will work to create graduate-postdoc teams that are mutually beneficially. Mentorship specifics will depend on the project, but may include:

\begin{enumerate}
\item Training the graduate student on protocols and methods.
\item Working with the graduate student to either collect or analyze data of mutual interest.
\item Co-writing manuscript outlines and drafts with the graduate student.
\end{enumerate}

At times graduate student mentorship will involve including them on your own projects (often to the point of authorship), at other times it will involve them including you on theirs. In the latter case, it would be most typical for the postdoc to be listed as second author on resulting manuscripts, with Lauren's letters of recommendation indicating that the postdoc played a senior mentorship role on the publication.

\item Contribute to assigned lab tasks (keeping website up-to-date, tidying up, maintaining the listserv or lab meeting schedule, etc.)

\item Attend research conferences. Most grants and fellowships have money earmarked for travel, and so the lab will attempt to cover lab-related travel for which you are the primary author and presenter. That said, money saved on grants can be used to forward your research, and so you are expected to apply for travel grants.
  
\item Be present during the academic school year, except for fieldwork, conference travel, etc. Lauren will not keep track of when you are in the lab or office, but you are expected to be present enough for the casual interactions and discussions that fuel collaboration with your labmates and colleagues.  

The postdoc is a transitional period of time. Lauren will work with you to find a balance between personal needs and professional progress. If you anticipate a reason to be away from the lab for an extended time, communicate with Lauren to work out a plan together. Note that flexibility here will depend on a number of factors, including the need, the available funding, and your ability to engage remotely.
  
If you are away for any reason (field work, personal time, etc) you are expected to note it on the lab calendar so that folks are not wondering where you are. 

\item Follow lab protocols such as equipment checkout, data and project management, lab safety, etc.

\end{enumerate}


\section{Undergraduates - What Lauren expects of you}
\begin{enumerate}
\item Be an engaged member of the laboratory and contribute to an inclusive and collegial environment. This includes:
\begin{enumerate}
\item Attending lab events, such as lab meetings, when possible, and communicating when you have conflicts (e.g., we try to schedule events to avoid class time, but it's not generally possible to accomodate everyone)
\item Paying attention and taking good notes when learning protocols and methods, and paying it forward in terms of teaching others 
\item Engaging with other undergraduates and providing feedback on each other's experimental design, proposals, talks, and drafts
\item Treating one another with courtesy and respect
\end{enumerate}

\item When new to the lab, undergraduate work will initially involve duties that support projects led by graduate students and postdocs. We expect at least two quarters of dedicated work in supporting roles that demonstrates of commitment, enthusiasm and attention to detail, prior to independent project development. Subsequent to this period, thesis students generally either take ownership of a component of their grad/postdoc mentor's research program, take ownership of a component of a collaborative undergraduate research program, or develop their own projects that support the interests of the lab. 

\item All students are expected to understand the ``why'' as well as the ``what'' of what they are doing - this involves actively asking your grad/postdoc mentor questions when you are curious or confused, reading the literature they recommend, seeking out additional relevant literature and discussing what you've read with your mentor.

\item All students are expected to faithfully commit to a schedule they build with their grad/postdoc mentor, and be available and willing to contribute beyond the schedule in emergencies.

\item All students are expected to be proactive and communicative about deadlines and scheduling conflicts as they arise.

\item Thesis students are expected to build your own research ideas in collaboration with your grad/postdoc mentor; this involves spending time with the literature, deep thinking and brainstorming, and being receptive and responsive to feedback from your mentor, Lauren, and the lab.

\item Thesis students are expected to make an effort to fund your own research by seeking out fellowship and grant opportunities, responding when Lauren or your grad/postdoc mentor directs you to funding opportunities, and putting care and attention into proposals. Common funding opportunities include UROP, SPUR, UnderGrEBES, and the O'Day fellowship. 

\item Thesis students are expected to enroll for formal thesis requirements in their program (most commonly Biology, ESCI, ENVS, or the Honors College), to stay on top of relevant deadlines, and to complete an official thesis. Exceptions may be granted if you *cannot* formally complete a thesis due to credit hour conflicts; if this may be an issue communicate with us in a timely way.

\end{enumerate}

\end{document}


\section{Lab Technicians}
\begin{enumerate}
\item is expected to be a leader in the laboratory, helping to foster
  an atmosphere of inclusively, exploration, and learning
\item faithfully commit to the contracted work hours
\item be flexible and available and willing to contribute beyond
  scheduled hours in emergencies or crunch times; hours will be
  compensated during non-critical periods
\item be proactive in troubleshooting and technology development
\item contribute to training new lab personnel
\item may take opportunities to work on self-inspired or self-motivated
  projects, after assigned duties are fulfilled.
\item follow priorities set by PI or lab, keep track of hours spent on
  tasks
\item Inform the PI when planning to take vacation days a month in
  advance for multi-day vacations, a week in advance for a single
  vacation day (i.e., a Friday).
\item Sick days are for when you are sick, or need an extended
  doctor's visit. Call the PI the day you wake up sick and inform
  them that you are sick. For regular doctors
  visits up to four hours can be taken. Sick days cannot be taken
  with vacation days unless you become sick before a planned
  vacation. 
\end{enumerate}


%%% Local Variables:
%%% mode: latex
%%% TeX-PDF-mode: t
%%% End:


